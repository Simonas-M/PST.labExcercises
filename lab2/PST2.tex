%Testavimo planas pagal IEEE 829-1998

\section{Testavimo plano identifikatorius}

Šio testavimo plano unikalus ID: PST.TP.1.  Versija - 1.1.

Plano autorius: 
Simonas Mikulis (Programų Sistemos, 3 kursas 5 grupė)
Mantas Petrikas (Programų Sistemos, 3 kursas 5 grupė)

Kontaktinė informacija:
Elektroninis paštas: simonas.mikulis@mif.stud.vu.lt
Elektroninis paštas: mantas.petrikas@mif.stud.vu.lt


\section{Dokumentai ir nuorodos į juos}

Užduoties pobūdis:

Užduočių valdymo sistema, skirta planuoti ir vykdyti užduotis vadovaujantis 'Agile' metodologija
Dokumento versija: 1.0

Agile principų specifikacija
http://www.agilemanifesto.org
Dokumento versija: 1.0

Kanban 'Lean' metodo specifikacija
http://www.everydaykanban,com/what-is-kanban
Dokumento versija: 1.0


\section{Įžanga}

Šio darbo tikslas sudaryti planą užduočių valdymos sistemos veikimo testavimui atlikti.
Pagal šį planą ištestuoti programinę įrangą ir užfiksuoti rastus defektus.
Plano lygmuo - „Master“. Planas apima pasirinktos programos testavimą, atitikimą reikalavimams. 
Biudžeto šiai sistemai nėra, resursai – du žmonės, programos kūrėjai yra šio plano autoriai. 
Testavimas pagal šį planą bus vykdomas nuo sistemos kūrimo pradžios. Sistemai sukurti bus skirta 20 iteracijų po 2 savaites.


\section{Testavimo objektai}

Bus testuojama tai kas pateikiama vartotojui naudotis:

\begin{itemize}
	\item Vartotojo sąsaja
	\item Sistemos vientisumas naudojantis keliems vartotojams
	\item Klaidu apdorojimas
	\item Našumas
\end{itemize}


\section{Programinės įrangos rizikos}

Kadangi užduočių valdymo sistema nėra unikalus produktas, reikės sukurti programinę įrangą,
kuri savo funkcionalumu bei intuityvia vartotojo sąsaja nukonkuruotų varžoves.
Išorinės integracijos, kurios bus naudojamos užduočių valdymo sistemoje, yra nusistovėjusios, todėl labai didelės
rizikos nėra, tačiau sistemoje bus sukurtas integracijų plėtimo taškai lengvam naujovių įdiegimui.
Naudotojų duomenys bus saugomi duomenų bazėje, todėl yra nedidelė, bet egzistuojanti rizika, kad duomenys gali būti
prarasti arba nutekinti tretiesiems asmenims.


\section{Testuojamos savybės}

Pagal defektų atsiradimo tikimybę programinės įrangos veikimo metu, rizikos lygmenys yra standartiškai suskirstyti į:

\begin{itemize}
	\item Žemas (Ž)
	\item Vidutinis (V)
	\item Aukštas (A)
\end{itemize}

Iš naudoto pusės bus testuojama:

\begin{itemize}
	\item Greitaveika (lygis - V)
	\item Vartotojo sąsajos aiškumas (lygis - A)
	\item Sistemos patikimumas (lygis - A)
	\item Vartotojo sąsajos dizainos (lygis - Ž)
\end{itemize}


\section{Netestuojamos savybės}

Iš vartotojo pusės nebus testuojamos senesnės programinės įrangos pvz.: Interneto naršyklių, SSH klientų palaikymas.
Tai bus daroma dėl to, kad kuriama sistema atitktų naujausius standartus bei gerasias praktikas.
Tuo pačiu nebus švaistomas brangus programos kūrimo laikas, kuris bus panaudotas programinės įrangos kokybės užtikrinimui


\section{Taktika / Priėjimas}

Pagal 'Agile' metodologiją, nustatoma, kad sprintas truks 2 savaites.
Kiekvieno sprinto pabaigoje bus įvykdomas demo ir retrospektyvos,
 kurių metu bus aptariamos praėjusio sprinto teigiami ir neigiami aspektai.
Programinės įrangos kurimo procese bus naudojama TDD praktika - pra
Kadangi prie sistemos dirbs tik du žmonės kasdieniniai susitikimai bus pakeisti pokalbiais prie kavos.
Už programos testų rašymą bus atsakingi programuotojai.


Sistemos kūrimui : IntelliJ IDEA Ultimate, Git, MongoDB, JavaScript, Windows 10 bei Ubuntu operacines sistemas.
Įrankiai gerai žinomi ir išmanomi, todėl papildomų mokymų nereikės. 

Programinės įrangos kurimo procese bus renkamos šios metrikos: fuckcijų sudėtingumas, kodo padengimas testais.
Sistemos konfiguracija bus valdoma konfiguraciniais failais,
 kūrimo, testavimo ir produkcinėms aplinkoms bus sukūrtos atskiros konfiguracijos, bus testuojama visose aplinkose.
Sukurta sistema bus testuojama šiomis naršyklėmis: Mozilla Firefox, Google Chrome, Safari.
Išleidus naują programos versiją bus atliekami regresiniai testai sistemos našumui,
 patikimumui ir ankstesnės sistemos versijos funkcionalumui.
Naujam funkcionalumui testuoti bus rašomi vienetų testai. 

Defektų valdymo proceso aprašas:

Testavimo metu randami defektai bus kaupiami ir registruojami. 
Kiekvienam defektui suteikiamas unikalus ID, jo kilimui nustatyti ir taisyti bus sudaromas defekto aprašas, poveikio lygmuo, 
 testavimo atvejo ID, kurio metu defektas užfiksuotas, bei norima programos elgsena.

Pagal poveikį programinės įrangos veikimui, defektai yra suskirstyti į:

\begin{itemize}
	\item Svarbus - defektas trukdo programai teisingai atlikti esmines funkcijas. Kilus šio tipo programos
	veikimo sutrikimui, programa dalimi atvejų veiks kitaip negu numatyta reikalavimuose,
	gali susigadinti dalis programos išsaugotų duomenų.
	\item Vidutinis - defektas trukdo programai teisingai atlikti pagalbines funkcijas. Kilus šio tipo
	programos veikimo sutrikimui, programa dalimi atvejų gali veikti neefektyviai, lėčiau negu turėtų arba užduotį galima įgyvendinti kitu scenarijumi.
  \item Žemas - defektas netrukdo vartotojui atlikti veiksmų, bet neatitinka programinės įrangos ar vartotojo lukesčių speficikacijos.
\end{itemize}

\section{Sėkmingo / Nesėkmingo testavimo kriterijai}

Testavimo atvejis yra laikomas sėkmingu, kai jo vykdymo metu nekyla jokie defektai, tikimasis atvejo rezultatas sutampa su atvejo vykdymo rezultatu. 
Kitu atveju, laikoma nesėkmingu, fiksuojamas defektas.

Testavimo etapas laikomas sėkmingu, kai yra įvykdomi visi (100\%) testavimo atvejai, 80\% kodo padengta testais. 

Projekto testavimas laikomas sėkmingu, kai yra pilnai įvykdyti visų etapų testavimo planai, neištaisytų defektų yra ne daugiau kaip 10\% ir jų kritiškumas nėra svarbus. 
Kitu atveju, laikoma nesėkmingu, toliau šalinami aptikti defektai, vykdomi etapų testavimo planai.

\section{Sustabdymo ir pratęsimo kriterijai}

Stabdymas daromas, jei defektų kiekis testavimo etapo metu viršija 30\% lyginant su testavimo atvejų kiekiu. 
Testavimas pratesiamas jei defektų kiekis neviršyja 10\% visų testavimo atvejų.

\section{Planuojami darbo produktai}

Darbo metu planuojami: 

\begin{itemize}
	\item Sudarytas testavimo planas
	\item Sukurti testavimo atvejai, padengiantys bent 90\% kodo ir visus testavimo objektus
	\item Testavimo atvejų vykdymo, defektų aptikimo ir jų šalinimo rekomendacijų analizės žurnalas
\end{itemize}

\section{Likusios testavimo užduotys}

Testavimą vykdo 2 asmenys, kurie ir kuria testavimo planą bei sistemą, todėl likusias užduotis jie ir vykdys:

\begin{itemize}
	\item Sukurti testavimo atvejus
	\item Įgyvendinti testavimo atvejus
	\item Užfiksuoti ir aprašyti aptiktus defektus
	\item Aprašyti defektų šalinimo rekomendacijas
\end{itemize}

\section{Reikalingos aplinkos}

Testavimui atlikti reikalingos aplinkos yra: operacinė aplinka: Windows 10, MacOS arba Linux destribucija, naršyklė Mozilla Firefox, Google Chrome arba Safari. 
Testavimui skirti failai pasiekiami per Git versijavimo sistemą. 

\section{Kompetencijų ir mokymų poreikiai}

Sistemos kūrimui ir testavimui atlikti reikalinga žinoti ir mokėti dirbti su Java 9, Spring Framework, JUnit,
 MongoDB, JavaScript es6, React 16, Karma, Mocha, Redux, React-router, Postman, Apache JMeter, Selenium Web Driver. 
Testuojamos sistemos ir įrankių naudojimo mokymai nebus organizuojami, individualiam tobulėjumui galima naudoti PluralSight.

\section{Atsakomybės}

Kadangi prie sistemos dirba vienas žmogus, jis ir yra atsakingas už visus pateikiamus kriterijus: 

\begin{itemize}
	\item Rizikų valdymą
	\item Testuojamas ir netestuojamas savybes
	\item Testavimo taktiką
	\item Testavimo etapų nustatymą ir aprašymą
	\item Testavimo atvejų sukūrimą
	\item Aplinkas
	\item Grafiko problemų sprendimą
	\item Kompetencijas, žinių turėjimą
	\item Kritinius sprendimus, neapibrėžtus testavimo plane
	\item Testavimo rezultatų rinkimą ir registravimą
\end{itemize}

\section{Grafikas}

Testavimo plano vykdymo grafikas yra pateikiamas pagrįstas realistiškais ir patikrintais įvertinimais.
Darbo vykdymo laikas nurodomas ne konkrečiomis datomis, o susietas su programavimo, testavimo rezultatais ir prieš tai esančių žingsnių vykdymu.

\begin{itemize}
	\item Iš dėstytojo gautų, pradinių reikalavimų peržiūra.
	\item Reikalavimų specifikacijos sudarymas, prasideda valanda po pradinių reikalavimų peržiūros.
	\item „Master“ lygmens testavimo plano sudarymas, diena po reikalavimų specifikacijos sudarymo.
	\item Testavimo etapų planavimas ir valdymas, valanda po pagrindinio testavimo plano sudarymo.
	\item Testavimo atvejų kūrimas ir analizė, prasideda diena po visų testavimo planų sudarymo.
	\item Vienetų testavimo vykdymas, daromas sistemos surikimo metu, diena po to, kai paruošiami testavimo atvejai.
	\item Integracijos, sistemos ir priėmimo testavimo vykdymas, diena po sistemos surinkimo pabaigos.
	\item Rezultatų rinkimas, registravimas ir kaupimas, vykdomas testavimo metu.
	\item Rezultatų vertinimas ir ataskaitų rengimas, prasideda valanda po rezultatų surinkimo pabaigos.
\end{itemize}

Nukrypus nuo grafiko spartinimas grafiko vykdymas, žingsniai kurie turėtų būti pradedami vykdyti diena po prieš tai esančio žingsnio pabaigimo, pradeda tą pačią dieną. 
Stengiamasi išlaikyti nukrypimą kuo mažesnį, bei nenukrypti nuo grafiko dar labiau su tolimesnių testavimo proceso žingsnių vykdymu. 
Jeigu nukrypimas  nuo grafiko vis dar yra likus savaitei iki atsiskaitymo datos, atsižvelgiama į „Stabdymo ir pratęsimo“ metu nurodytus lengvatinius kriterijus sistemos galutiniam ištestavimui.

\section{Rizikos ir nenumatyti atvejai}

Jeigu trūks žmonių ar mokymų, nieko pakeisti neįmanoma. Darbą atlieka vienas žmogus, todėl papildomai prie darbo su sistema pakviesti negalima ir vykdyti mokymų neapsimoka. 

Jeigu trūks įrankių, duomenų ar žinių likus savaitei iki atsiskaitymo datos, bus sumažintas atliekamų testų skaičius, prailginamos darbo valandos, kad įgyti papildomų žinių, gauti reikiamų duomenų ir įrankių.

Jeigu vėluos programavimas, likus savaitei iki atsiskaitymo datos, bus sumažintas atliekamų testų skaičius, padidintas priimtinas defektų skaičius, prailginamos darbo valandos. Viso to būtų imamasi, kad visi darbai būtų pilnai užbaigti iki atsiskaitymo datos.

Jeigu keisis reikalavimai, likus savaitei iki atsiskaitymo datos, nukelta testavimo pabaigos data viena diena tolyn, padidintas priimtinas defektų skaičius, prailginamos darbo valandos. Viso to būtų imamasi, kad iki atsiskaitymo datos, būtų užtikrintas atitikimas visiems pasikeitusiems reikalavimams.

\section{Patvirtinimai}

Testavimo plano korektiškumą patvirtins jo sudarymui vadovaujantis dėstytojas (darbo vadovas).

\section{Žodynėlis}

Sąvokos:

\begin{itemize}
	\item HTTP 1.1 Standartas - standartinis būdas informacijai pasauliniame internetiniame tinkle (WWW) pasiekti. 
		  Pradinė protokolo paskirtis – pateikti standartinį būdą HTML puslapiams skelbti ir skaityti.
	\item Pasaulinis internetinis tinklas (WWW) - interneto dalis, resursai, kuriuos internete galima pasiekti naudojant URL (Vieningus Resursų Identifikatorius).
		  Pasaulinis tinklas daugiausia remiasi hipertekstu – HTTP protokolu ir HTML kalba.
	\item Hiperteksto žymėjimo kalba (HTML) - tai kompiuterinė žymėjimo kalba, naudojama pateikti turinį internete.
	\item Dokumento adreso lokatorius (URL) - tai unikalus adresas, nurodantis kur yra patalpintas internetinis puslapis ir kaip jį pasiekti.
	\item ,,Master'' lygmuo - Testavimo plano lygmuo, nurodantis, kaip bus vykdomas visas testavimo planas bei kad bus kuriami testavimo etapų planai.
	
\end{itemize}

Sutrumpinimai:

\begin{itemize}
	\item HTTP - trumpinys iš angl. HyperText Transfer Protocol
	\item WWW - trumpinys iš angl. World Wide Web
	\item HTML - trumpinys iš angl. Hyper text Markup Language
	\item URL - trumpinys iš angl. Uniform Resource Locator
	\item Testavimo lygmenys:
		\begin{itemize}
			\item V - vienetų
			\item I - integracijos
			\item S - sistemos
			\item P - priėmimo
		\end{itemize}
	\item Rizikų lygmenys:
		\begin{itemize}
			\item Ž - žemas
			\item V - vidutinis
			\item A - aukštas
		\end{itemize}
\end{itemize}
