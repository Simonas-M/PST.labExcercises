%Testavimo planas pagal IEEE 829-1998

\documentclass{VUMIFPSkursinis}
  \usepackage{algorithmicx}
    \usepackage{algorithm}
    \usepackage{algpseudocode}
    \usepackage{amsfonts}
    \usepackage{amsmath}
    \usepackage{bm}
    \usepackage{caption}
    \usepackage{color}
    \usepackage{float}
    \usepackage{graphicx}
    \usepackage{listings}
    \usepackage{subfig}
    \usepackage{wrapfig}
    
    % Titulinio aprašas
    \university{Vilniaus universitetas}
    \faculty{Matematikos ir informatikos fakultetas}
    \department{Programų sistemų katedra}
    \papertype{Testavimo planas}
    \title{Programų sistemų kūrimo metodų tyrimas}
    \status{3 kurso 5 grupės studentai}
    \author{Simonas Mikulis}
    \secondauthor{Mantas Petrikas}   % Pridėti antrą autorių
    \supervisor{Dr. Vytautas Valaitis}
    \date{Vilnius – \the\year}
    
    % Nustatymai
    \setmainfont{Palemonas}   % Pakeisti teksto šriftą į Palemonas (turi būti įdiegtas sistemoje)
    \bibliography{bibliografija}
    
    \begin{document}
      \maketitle
    
      \tableofcontents
  
    \section{Testavimo plano identifikatorius}
    
    Šio testavimo plano unikalus ID: PST.TP.1.  Versija - 1.1.
    
    Plano autorius: 
    Simonas Mikulis (Programų Sistemos, 3 kursas 5 grupė)
    Mantas Petrikas (Programų Sistemos, 3 kursas 5 grupė)
    
    Kontaktinė informacija:
    Elektroninis paštas: simonas.mikulis@mif.stud.vu.lt
    Elektroninis paštas: mantas.petrikas@mif.stud.vu.lt
    
    
    \section{Dokumentai ir nuorodos į juos}
    
    Užduoties pobūdis:
    
    Užduočių valdymo sistema, skirta planuoti ir vykdyti užduotis vadovaujantis „Agile“ metodika
    Dokumento versija: 1.0
    
    Agile principų specifikacija
    http://www.agilemanifesto.org
    Dokumento versija: 1.0
    
    Kanban „Lean“ metodo specifikacija
    http://www.everydaykanban,com/what-is-kanban
    Dokumento versija: 1.0
    
    
    \section{Įžanga}
    
    Šio darbo tikslas sudaryti planą užduočių valdymos sistemos veikimo testavimui atlikti.
    Pagal šį planą ištestuoti programinę įrangą ir užfiksuoti rastus defektus.
    Plano lygmuo - „Master“. Planas apima pasirinktos programos testavimą, atitikimą reikalavimams. 
    Biudžeto šiai sistemai nėra, resursai – du žmonės, programos kūrėjai yra šio plano autoriai. 
    Testavimas pagal šį planą bus vykdomas nuo sistemos kūrimo pradžios. Sistemai sukurti bus skirta 20 iteracijų po 2 savaites.
    
    
    \section{Testavimo objektai}
    
    Bus testuojama tai kas pateikiama vartotojui naudotis:
    
    \begin{itemize}
    	\item Vartotojo sąsaja
    	\item Sistemos vientisumas naudojantis keliems vartotojams
    	\item Klaidų apdorojimas
    	\item Našumas
    \end{itemize}


    \section{Programinės įrangos rizikos}

    Kadangi užduočių valdymo sistema nėra unikalus produktas, reikės sukurti programinę įrangą,
     kuri savo funkcionalumu bei intuityvia vartotojo sąsaja nukonkuruotų varžoves.
    Išorinės integracijos, kurios bus naudojamos užduočių valdymo sistemoje, yra nusistovėjusios, todėl labai didelės
     rizikos nėra, tačiau sistemoje bus sukurti integracijų plėtimo taškai lengvam naujovių įdiegimui.
    Naudotojų duomenys bus saugomi duomenų bazėje, todėl yra nedidelė, bet egzistuojanti rizika, kad duomenys gali būti
     prarasti arba nutekinti tretiesiems asmenims.


    \section{Testuojamos savybės}

    Pagal defektų atsiradimo tikimybę programinės įrangos veikimo metu, rizikos lygmenys yra standartiškai suskirstyti į:

    \begin{itemize}
    	\item Žemas (Ž)
    	\item Vidutinis (V)
    	\item Aukštas (A)
    \end{itemize}

    Iš naudotojo pusės bus testuojama:

    \begin{itemize}
    	\item Greitaveika (lygis - V)
    	\item Vartotojo sąsajos aiškumas (lygis - A)
    	\item Sistemos patikimumas (lygis - A)
    	\item Vartotojo sąsajos dizainos (lygis - Ž)
    \end{itemize}


    \section{Netestuojamos savybės}

    Iš vartotojo pusės nebus testuojamos senesnės programinės įrangos pvz.: Interneto naršyklių, SSH klientų palaikymas.
    Tai bus daroma dėl to, kad kuriama sistema atitktų naujausius standartus bei gerąsias praktikas.
    Tuo pačiu nebus švaistomas brangus programos kūrimo laikas, kuris bus panaudotas programinės įrangos kokybės užtikrinimui


    \section{Taktika / Priėjimas}

    Pagal „Agile“ metodiką, nustatoma, kad sprintas truks 2 savaites.
    Kiekvieno sprinto pabaigoje bus įvykdomas demo ir retrospektyvos,
     kurių metu bus aptariamos praėjusio sprinto teigiami ir neigiami aspektai.
    Programinės įrangos kurimo procese bus naudojama TDD praktika.
    Kadangi prie sistemos dirbs tik du žmonės kasdieniniai susitikimai bus pakeisti pokalbiais prie kavos.
    Už programos testų rašymą bus atsakingi programuotojai.


    Sistemos kūrimui naudotina programinė įranga: IntelliJ IDEA Ultimate, Git, MongoDB, JavaScript, Windows 10 bei Ubuntu operacinės sistemos.
    Įrankiai sistemoms kūrėjams gerai žinomi ir išmanomi, todėl papildomų mokymų nereikės. 

    Programinės įrangos kurimo procese bus renkamos šios metrikos: funkcijų sudėtingumas, kodo padengimas testais.
    Sistemos konfiguracija bus valdoma konfiguraciniais failais.
	Kūrimo, testavimo ir produkcinėms aplinkoms bus sukūrtos atskiros konfiguracijos, bus testuojama visose aplinkose.
    Sukurta sistema bus testuojama šiomis naršyklėmis: Mozilla Firefox, Google Chrome, Safari.
    Išleidus naują programos versiją bus atliekami regresiniai testai sistemos našumui,
     patikimumui ir ankstesnės sistemos versijos funkcionalumui.
    Naujam funkcionalumui testuoti bus rašomi vienetų testai. 

    Defektų valdymo proceso aprašas:

    Testavimo metu randami defektai bus kaupiami ir registruojami. 
    Kiekvienam defektui suteikiamas unikalus ID, jo kilimui nustatyti ir taisyti bus sudaromas defekto aprašas, poveikio lygmuo, 
     testavimo atvejo ID, kurio metu defektas užfiksuotas, bei norima programos elgsena.

    Pagal poveikį programinės įrangos veikimui, defektai yra suskirstyti į:

    \begin{itemize}
    	\item Svarbus - defektas trukdo programai teisingai atlikti esmines funkcijas. Kilus šio tipo programos
    	veikimo sutrikimui, programa dalimi atvejų veiks kitaip negu numatyta reikalavimuose,
    	gali susigadinti dalis programos išsaugotų duomenų.
    	\item Vidutinis - defektas trukdo programai teisingai atlikti pagalbines funkcijas. Kilus šio tipo
    	programos veikimo sutrikimui, programa dalimi atvejų gali veikti neefektyviai, lėčiau negu turėtų arba užduotį galima įgyvendinti kitu scenarijumi.
      \item Žemas - defektas netrukdo vartotojui atlikti veiksmų, bet neatitinka programinės įrangos ar vartotojo lūkesčių speficikacijos.
    \end{itemize}

    \section{Sėkmingo / Nesėkmingo testavimo kriterijai}

    Testavimo atvejis yra laikomas sėkmingu, kai jo vykdymo metu nekyla jokie defektai, tikimasis atvejo rezultatas sutampa su atvejo vykdymo rezultatu. 
    Kitu atveju, laikoma nesėkmingu, fiksuojamas defektas.

    Testavimo etapas laikomas sėkmingu, kai yra įvykdomi visi (100\%) testavimo atvejai, 80\% kodo padengta testais. 

    Projekto testavimas laikomas sėkmingu, kai yra pilnai įvykdyti visų etapų testavimo planai, neištaisytų defektų yra ne daugiau kaip 10\% ir jų kritiškumas nėra svarbus. 
    Kitu atveju, laikoma nesėkmingu, toliau šalinami aptikti defektai, vykdomi etapų testavimo planai.

    \section{Sustabdymo ir pratęsimo kriterijai}

    Stabdymas daromas, jei defektų kiekis testavimo etapo metu viršija 30\% lyginant su testavimo atvejų kiekiu. 
    Testavimas pratesiamas jei defektų kiekis neviršyja 10\% visų testavimo atvejų.

    \section{Planuojami darbo produktai}

    Darbo metu planuojami: 

    \begin{itemize}
    	\item Sudarytas testavimo planas
    	\item Sukurti testavimo atvejai, padengiantys bent 80\% kodo ir visus testavimo objektus.
    	\item Testavimo atvejų vykdymo, defektų aptikimo ir jų šalinimo rekomendacijų analizės žurnalas
    \end{itemize}

    \section{Likusios testavimo užduotys}

    Testavimą vykdo 2 asmenys, kurie ir kuria testavimo planą bei sistemą, todėl likusias užduotis jie ir vykdys:

    \begin{itemize}
    	\item Sukurti testavimo atvejus
    	\item Įgyvendinti testavimo atvejus
    	\item Užfiksuoti ir aprašyti aptiktus defektus
    	\item Aprašyti defektų šalinimo rekomendacijas ir sistemos elgseną pašalinus defektus
    \end{itemize}

    \section{Reikalingos aplinkos}

    Testavimui atlikti reikalingos aplinkos yra: operacinė aplinka: Windows 10, MacOS arba Linux destribucija, naršyklė Mozilla Firefox, Google Chrome arba Safari. 
    Testavimui skirti failai pasiekiami per Git versijavimo sistemą. 

    \section{Kompetencijų ir mokymų poreikiai}

    Sistemos kūrimui ir testavimui atlikti reikalinga žinoti ir mokėti dirbti su Java 9, Spring Framework, JUnit,
     MongoDB, JavaScript ES6, React 16, Karma, Mocha, Redux, React-router, Postman, Apache JMeter, Selenium Webdriver. 
    Testuojamos sistemos ir įrankių naudojimo mokymai nebus organizuojami, nes jiems organizuoti nepakanka lėšų, individualiam tobulėjumui galima naudoti PluralSight.

    \section{Atsakomybės}

    Prie sistemos kūrimo ir testavimo prisidės du žmonės, jie bus atsakingi už visus pateikiamus kriterijus: 

    \begin{itemize}
    	\item Rizikų valdymą
    	\item Testuojamų ir netestuojamų savybių parinkimą
    	\item Testavimo taktiką
    	\item Testavimo etapų nustatymą ir aprašymą
    	\item Testavimo atvejų sukūrimą
    	\item Aplinkų paruošimą testavimui
      	\item Grafiko problemų sprendimą
    	\item Kritinius sprendimus, neapibrėžtus testavimo plane
    	\item Testavimo rezultatų rinkimą ir registravimą
    \end{itemize}

    \section{Grafikas}

    Kadangi sistemos kurimui numatytos 40 savaičių, naujų funkcijų kūrimas turėtų pasibaigti 36 savaitę.
    Likusios savaitės bus skirtos galutiniam sistemos testavimui ir defektų šalinimui. 
    Jeigu kritinės funkcijos nebus pabaigtos iki funkcijų kūrimo pabaigos, 2 savaitės gali būti skirtos šių funckijų užbaigimui,
     lygiagrečiai vykdant sistemos testavimą.
    Anksčiu nei suplanuota pabaigus programavimo darbus, galutinį testavimą galima pradėti anksčiau.
    
    \begin{itemize}
    	\item Iš užsakovo gautų pradinių reikalavimų peržiūra.
    	\item Detaliosios reikalavimų specifikacijos sudarymas, pradedamas ne vėliau kaip diena po pradinių reikalavimų peržiūros.
    	\item „Master“ lygmens testavimo plano sudarymas, pradedamas ne vėliau kaip po reikalavimų specifikacijos sudarymo.
    	\item Testavimo etapų planavimas ir specifikavimas, pradedamas po pagrindinio testavimo plano sudarymo.
    	\item Testavimo atvejų kūrimas ir analizė, pradedamas ne vėliau kaip diena po visų testavimo planų sudarymo.
    	\item Vienetų testavimo vykdymas, daromas sistemos surikimo metu ir po kiekvienos programavimo iteracijos.
    	\item Integracijos, sistemos ir priėmimo testavimo vykdymas, diena po sistemos surinkimo pabaigos.
    	\item Rezultatų rinkimas, registravimas ir kaupimas, vykdomas testavimo metu.
    	\item Rezultatų vertinimas ir ataskaitų rengimas, pradedamas po rezultatų surinkimo pabaigos.
    \end{itemize}

    Nukrypus nuo grafiko spartinimas grafiko vykdymas. Žingsniai, kurie turėtų būti pradedami po vėluojančio procesų pabaigos, pagal galimybes pradedami daryti lygiagrečiai. 
    Stengiamasi išlaikyti nukrypimą kuo mažesnį, bei nenukrypti nuo grafiko dar labiau su tolimesnių testavimo proceso žingsnių vykdymu. 
    
    \section{Rizikos ir nenumatyti atvejai}

    Jeigu dėl nenumatytų aplinkymių trūks ar nebus žmonių gebančių atlikti sistemos kūrimo ar testavimo darbus, 
     suderinus su užsakovu paliekama galimybė darbus sustabdyti ar atidėti, kol bus gauta pakankamai žmogiškųjų išteklių sistemai pabaigti. 

    Jeigu trūks įrankių ar duomenų testi sistemos kūrimo ar testavimo darbus ir neturint galimybės per 2 savaites gauti visus reikiamas resursus,
     gali būti mažinama sistemos apimtis, sistemoje paliekant plėtimo galimybes pilnam reikalavimų įgyvendinimui vėlesnėsę iteracijose, 
     kai bus gauta pakankamai resursų pilnai įgyvendinti reikalavimus.

    Jeigu keisis reikalavimai jau suderintoms projekto dalims, priklausomai nuo pakeitimų dydžio, gali būti atsisakyta dalies dar neįgyvendintų funckijų.
     Tai daroma norint užtikrinti kad darbai bus pabaigti laiku ir būtų užtikrintas sistemos atitikimas visiems pasikeitusiems reikalavimams.

    \section{Patvirtinimai}
    Projekto korektiškumą ir išbaigtumą kiekvieno sprinto pabaigoje patvirtins programuotojų ir testuotojų susitikimas. 
    Pagal programos kurėjų ir užsakovų poreikius, bet ne rečiau kaip kas mėnesį, bus organizuojami sistemos funkcionalumo pristatymai 
     dalyvaujant užsakovams ir sistemos kūrėjams, kurių metu bus verfikuojamas ir validuojamas sukurtas funkcionalumas, derinami pakeitimai.

    \section{Žodynėlis}

    Sąvokos:
    \begin{itemize}
	\item „Master“ lygmuo - Testavimo plano lygmuo, nurodantis, kaip bus vykdomas visas testavimo planas bei kad bus kuriami testavimo etapų planai.
      \item „Judrioji“ metodika - programų sistemų kūrimo praktika, kai apibrėžtas produkto kūrimo kaina ir trukmė, bet dėl galimų reikalavimų pakitimų neapibrežtas rezultatas.
       Paprastai produktas kuriamas ir diegimas etapais, kiekvieno metu prisitaikant prie paskitusių reikalavimų.
	  \item Sprintas - viena fiksuoto laiko darbų iteracija. 
	  \item „Lean“ metodika - sunaudojant kuo mažiau resursų, pateikti kuo geresnį produktą.
    \end{itemize}

    Sutrumpinimai:
    \begin{itemize}
    	\item Rizikų lygmenys:
    		\begin{itemize}
    			\item Ž - žemas
    			\item V - vidutinis
    			\item A - aukštas
    		\end{itemize}
    \end{itemize}

  \end{document}